\documentclass{article}

\usepackage{amsmath}
\usepackage{amssymb}
\usepackage{parskip}
\usepackage{fullpage}
\usepackage{hyperref}
\usepackage{graphicx}
\usepackage{xcolor}
\usepackage{tcolorbox}
\usepackage{changepage}
\usepackage{framed}
\usepackage{biblatex}
\usepackage{filecontents}
\usepackage[italian]{babel}

\usepackage{eso-pic}
\newcommand\BackgroundPic{
    \put(0,0){
        \parbox[b][\paperheight]{\paperwidth}{
            \vfill
            \centering
            \includegraphics[width=\paperwidth,height=\paperheight]{background}
            \vfill
        }
    }
}

\hypersetup{
    colorlinks=true,
    linkcolor=black,
    urlcolor=blue,
    pdftitle={Laovor PDI},
    pdfpagemode=FullScreen,
}

%\begin{filecontents}{references.bib}
%@online{paolobettelini,
%  author    = {Paolo Bettelini},
%  title     = {notes},
%  year      = {2022},
%  publisher = {GitHub},
%  journal   = {GitHub repository},
%  url       = {https://github.com/paolobettelini/notes}
%}
%\end{filecontents}

%\addbibresource{references.bib}
\graphicspath{ {./images/} }

\title{
    \textsc{Il ruolo dello Stato democratico nella società umana}
    \\
    {\small \textsc{La democrazia nel sistema giudiziario}}
}

\author{
    \textsc{Paolo Bettelini}
    \\
    \textsc{\large Scuola d'Arti e Mestieri di Trevano (SAMT)}
}

\date{\textsc{\today}}

\begin{document}

\AddToShipoutPicture*{\BackgroundPic}

\makeatletter
\renewcommand{\maketitle}{
    \vspace*{20pt}
    \begin{adjustwidth}{}{-1in}
        \begin{flushright}
            \begin{tcolorbox}[sharp corners,
                colback=white,
                colframe=white,
                text width=\dimexpr9cm+1in]
                
                \centering
                
                {\LARGE\@title}
                
                \vspace{30pt}
                
                {\large\@author}
                
                \@date
                
                \vspace{20pt}
            \end{tcolorbox}
        \end{flushright}
    \end{adjustwidth}

    \vspace{12cm}

    \begin{adjustwidth}{-1in}{}
        \begin{flushleft}
            \begin{tcolorbox}[sharp corners,
                colback=white,
                colframe=white,
                text width=\dimexpr5cm+1in]
                
                Progetto Didattico Interdisciplinare
                
                Classe I4AC 2022/2023
                
                Monica Delucchi, Ursula Holliger
            \end{tcolorbox}
        \end{flushleft}
    \end{adjustwidth}
}
\makeatother
\maketitle

\pagebreak

\tableofcontents
\pagebreak

\section{Introduzione}

Durante la mia vita mi sono ritrovato a sviluppare dei pensieri
spesso in discordia con la maggior parte delle popolazione.
Mediante delle successioni di pensieri logici ho scoperto autonomamente concetti
come il \textit{determinismo}, senza aver mai avuto nessun contatto con dei testi di filosofia.
\\
Questo lavoro è una riflessione personale che sviluppa i miei pensieri
personali circa la moralità delle cose. Verranno trattate le implicazioni
che questi concetti hanno sulla democrazia ed il sistema giudiziario.

Il documento tratterà la democrazia semplicemente come il concetto di integrare i cittadini
nel governo di una società. Non verranno discussi diversi modelli di democrazia o di stato.

Le prime sezioni del documento introducono quelli che sono i concetti fondamentali per
sviluppare il mio ragionamento. In seguito verranno trattate
le implicazioni di questi concetti sulla moralità di uno stato e come si potrebbero
ipoteticamente risolvere.

\section{Etica e Morale}

\subsection{Definizione}

\paragraph{Morale}
la morale è un insieme di pensieri, decisioni e comportamenti che secondo l'individuo
che li esercita sono \textit{corretti} nei confrinti di sè stesso o terze parti.

\paragraph{Etica}
L'etica è un insieme di pratiche, comportanenti e di norme che vengono applicati
per rispecchiare un'ideologia morale.

\subsection{Il regno animale}

\section{Il libero arbitrio}

Il libro arbitrio è un concetto per il quale gli esseri umani
(o in alcune eccezioni ogni essere cosciente) è in grado di decidere e agire secondo la propria volontà.
L'essere è quindi libero di arbitrare le proprie azioni secondo la sua volontà.

\section{Determinismo}

\subsection{Definizione}

Il determinismo è una corrente di pensiero che esprime l'idea
che gli eventi dell'universo siano completamente determinati e prevedibili.

\subsubsection{Determinismo duro}

Il determinismo duro è la versione più severa del determinismo.
Questo pensiero indica che tutti gli eventi siano strettamente
determinabili dagli eventi dell'universo precedenti.
Conoscendo l'esatto stato dell'universo in un dato istante \(t_n\), è possibile
determinare un qualsiasi istante futuro \(t_{m > n}\).

Questo ideologia è infatti in contrapposizione con il libero arbitrio.
Dal momento che tutti gli eventi nell'universo sono determinati da altri fattori determinati
o detemrinabili, i processi biologici che ci permettono di prendere delle decisioni e di pensare
sono soggette al determinismo duro, in quanto facenti parti di un sistema fisico esistente nell'universo.

\subsubsection{Compatibilismo}

Il compatibilismo è una corrente di pensieri che prevede appunto una compatibilità
fra il libero arbitrio ed il determinismo. Un'accezione generica di questa ideologia
implica che l'uomo abbia infatti il poterte di scegliere liberamente secondo la sua volontà.

\subsection{Determinismo nella fisica}

Il mondo in cui viviamo è governato dalla meccanica.
Tutte le nostre azioni quotidiane sono governate dalla meccanica.
Guidare l'auto, fare cadere un oggetto, camminare etc. sono tutte azioni
che rispettano delle leggi fisiche ben conosciute. \\
Agli inizi del XX secolo la comunità scientifica ha cominciato ad esplorare ciò che
è la meccanica quantistica. Queste nuovi leggi secondo le quali il nostro universo funziona
ad un livello microscopico, hanno generato diverse critiche verso il determinismo.

\subsubsection{Principio di Indeterminazione di Heisenberg}

Il principio di indeterminazione di Heisenberg è molto importante nelle discussioni
legale al determinismo e alla natura intrinseca dell'universo.
Questo concetto indica infatti che, considerando un oggetto quantistico (di dimensioni molto piccole),
l'ammontare di conoscenza sulla sua velocità è inversamente proporzionale all'ammontare
di conoscenza sulla sua posizione.
Se conosciamo esattamente la posizione di un oggetto, non sappiamo nulla sulla sua velocità.
Se conosciamo esattamente la sua velocità, non sappiamo nulla sulla posizione. Se siamo un po'
incerti sulla sua velocità, siamo anche un po' incerti sulla sua posizione.
La nostra conoscenza \textit{simultanea} di alcune proprietà è vincolata da delle leggi fisiche. \\
Questa affermazione può sembrare assurda e priva di senso, ma nel campo quantistico
alcune proprietà sono descritte da delle probabilità piuttosto che da dei valori sempre concreti.

Considerando questo fenomeno possiamo nel determinismo giungiamo ad una contraddizione.
Non è possibile conoscere lo stato dell'universo ad un certo istante se non possiamo
conoscere simultaneamente tutte le proprietà del nostro sistema.

\subsubsection{Eventi casuali}

La fisica quantistica è governata da funzioni probabilistiche.
Basti pensare semplicemente alla posizione di un elettrone. La posizione di un elettrone
di un sistema può essere descritta da una \textit{funzione d'onda}. Questa funzione descrive
la probabilità di misurare l'elettrone in un certo punto. Per esempio, considerando un elettrone libero
di trovarsi su una linea, potrebbe essere molto probabile che esso venga misurato essere nello spazio centrale, mentre
poco probabile che si trovi agli estremi.

\begin{figure}[h]
    \centering
    \includegraphics[width=0.5\textwidth]{wavefunction.jpg}
    \caption{Funzione d'onda}
\end{figure}

La natura deterministica dell'universo è dunque strettamente legata all'esistenza di
eventi puramente casuali.

\paragraph{Evento puramente casuale} un evento puramente casuale è un qualsiasi evento che potrebbe svolgersi in maniera differente se l'universo venisse riavvolto per ripetere nuovamente tale evento, senza che l'esecuzione dell'evento a priori abbia alcun effetto sulla nuova esecuzione.

Se nell'universo sono presenti eventi puramente casuali il determinismo non è corretto
in quanto non tutti gli avvenimento sono deducibili dagli eventi precedenti.

Gli eventi casuali presenti nella meccanica quantistica sono puramente casuali
oppure possono essere deducibili da altri fattori?
Questa è la domanda saliente circa il determinismo. La scienza non è mai riuscita a dimostrare
o sfatare nessuna delle due ipotesi.

\subsection{Determinismo nell'anatomia}

La seguente sezione tratta in maniera generale il funzionamento del cervello
e come prendiamo ogni decisione.

Il cervello è certamente una macchina molto complessa che nessuno
comprende pienamente. Tuttavia, la scienza riesce a spiegare a grandi linee
il sistema decisionale. Questo sistema si basa su tante piccole strutture cellulari
facenti parti del sistema neuroso, i \textit{neuroni}.

\begin{figure}[h]
    \centering
    \includegraphics[width=0.75\textwidth]{neuron.png}
    \caption{Struttura neuronale}
\end{figure}

[...]

Possiamo dunque dedurre che ogni decisione mai presa da ognuno di noi è
sostanzialmente dipendente da 3 fattori:
\begin{enumerate}
    \item Configurazione genetica iniziale
    \item Eventi vissuti
    \item Ambiente circostante
\end{enumerate}

\paragraph{Configurazione genetica iniziale}
Il modo di pensare è certamente diverso fra persona e persona.
Uno dei fattori molto importanti è la configurazione anatomica iniziale del cervello,
per cui la configurazione iniziale di sinapsi, neuroni etc. alla nascita.

\paragraph{Eventi vissuti}
Ogni evento vissuto può modificare la nostra capacità di pensare.
Un evento può consistere in un interazione fisica o semplicemente qualcosa di visto.
Durante la vita la configurazioni delle sinapsi evolve costantemente. Provare certe esperienze
o trovarsi in certe situazioni può certamente alterare la nostra capacità cognitiva.
Un evento può per esempio consistere all'alterazione fisica del cervello, come per esempio
un intervento di leucotomia oppure la malattia del morbo di Alzheimer.

\paragraph{Ambiente circostante}
Al momento di prendere una decisione l'ambiente circostante è uno dei principali fattori
principali. A seconda della situazioni, cerchiamo di prendere una decisione auspicabile
per affrontare quest'ultima.

\subsection{Determinismo nella filosofia}

\subsection{Determinismo nella politica}

\subsection{Implicazioni nella moralità}

Assumendo il determinismo si giunge ad un concetto che può essere molto pericoloso per una società
e le persone stesse. Questo concetto è la perdita del significato della colpa, del merito e altri simili.

\section{Il sistema giudiziario}

\subsection{Definizione e democrazia}

Il sistema giudiziario è un ente che si occupa di condannare gli atti illeciti.

\subsection{Funzionamento}

\subsection{Nella storia}

\subsubsection{La legge del taglione}


\subsubsection{La legge del contrappasso}
\subsubsection{La ghigliottina}
\subsubsection{Cesare Beccaria}

Cesare Beccaria (1738-1794) è stato un giurista e filosofo francese.
Nel 1764 pubblica "Dei delitti e delle pene" (cita) 
dove esamna una serie di difetti nel sistema dell'epoca che imponevano una giustizia equa.
Questa'opera è molto importante poiché delimita il peccatto (peccato religioso
che viene giudicato da Dio) da ciò che bisogna considerare da un punto di vista giuridico,
applicando leggi eque e uguali per tutti.

Cesare Beccaria era contrario alla pena di morte e alla tortura, infatti,
voleva dimostrare che queste metodologie fossero inefficaci e disumani.
Una delle sue idee primarie era l'importanza di prevenire un delitto piuttosto che punirlo.

\subsection{Moralità nel sistema giudiziario}

[punto principale del PDI] \\
Dati i ragionamenti logici nelle sezioni precedenti, è morale punire
un criminale?

\subsection{Implementazione alternativa}

\section{Conclusione}

\pagebreak

\listoffigures

\pagebreak

%\nocite{*} % cite all entries

%\printbibliography[heading=subbibliography]

\end{document}